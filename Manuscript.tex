% Options for packages loaded elsewhere
\PassOptionsToPackage{unicode}{hyperref}
\PassOptionsToPackage{hyphens}{url}
\PassOptionsToPackage{dvipsnames,svgnames,x11names}{xcolor}
%
\documentclass[
  12pt,
  a4paperpaper,
]{article}

\usepackage{amsmath,amssymb}
\usepackage{setspace}
\usepackage{iftex}
\ifPDFTeX
  \usepackage[T1]{fontenc}
  \usepackage[utf8]{inputenc}
  \usepackage{textcomp} % provide euro and other symbols
\else % if luatex or xetex
  \usepackage{unicode-math}
  \defaultfontfeatures{Scale=MatchLowercase}
  \defaultfontfeatures[\rmfamily]{Ligatures=TeX,Scale=1}
\fi
\usepackage{lmodern}
\ifPDFTeX\else  
    % xetex/luatex font selection
  \setmainfont[]{Garamond}
\fi
% Use upquote if available, for straight quotes in verbatim environments
\IfFileExists{upquote.sty}{\usepackage{upquote}}{}
\IfFileExists{microtype.sty}{% use microtype if available
  \usepackage[]{microtype}
  \UseMicrotypeSet[protrusion]{basicmath} % disable protrusion for tt fonts
}{}
\usepackage{xcolor}
\usepackage[margin=25mm]{geometry}
\setlength{\emergencystretch}{3em} % prevent overfull lines
\setcounter{secnumdepth}{-\maxdimen} % remove section numbering
% Make \paragraph and \subparagraph free-standing
\ifx\paragraph\undefined\else
  \let\oldparagraph\paragraph
  \renewcommand{\paragraph}[1]{\oldparagraph{#1}\mbox{}}
\fi
\ifx\subparagraph\undefined\else
  \let\oldsubparagraph\subparagraph
  \renewcommand{\subparagraph}[1]{\oldsubparagraph{#1}\mbox{}}
\fi


\providecommand{\tightlist}{%
  \setlength{\itemsep}{0pt}\setlength{\parskip}{0pt}}\usepackage{longtable,booktabs,array}
\usepackage{calc} % for calculating minipage widths
% Correct order of tables after \paragraph or \subparagraph
\usepackage{etoolbox}
\makeatletter
\patchcmd\longtable{\par}{\if@noskipsec\mbox{}\fi\par}{}{}
\makeatother
% Allow footnotes in longtable head/foot
\IfFileExists{footnotehyper.sty}{\usepackage{footnotehyper}}{\usepackage{footnote}}
\makesavenoteenv{longtable}
\usepackage{graphicx}
\makeatletter
\def\maxwidth{\ifdim\Gin@nat@width>\linewidth\linewidth\else\Gin@nat@width\fi}
\def\maxheight{\ifdim\Gin@nat@height>\textheight\textheight\else\Gin@nat@height\fi}
\makeatother
% Scale images if necessary, so that they will not overflow the page
% margins by default, and it is still possible to overwrite the defaults
% using explicit options in \includegraphics[width, height, ...]{}
\setkeys{Gin}{width=\maxwidth,height=\maxheight,keepaspectratio}
% Set default figure placement to htbp
\makeatletter
\def\fps@figure{htbp}
\makeatother
% definitions for citeproc citations
\NewDocumentCommand\citeproctext{}{}
\NewDocumentCommand\citeproc{mm}{%
  \begingroup\def\citeproctext{#2}\cite{#1}\endgroup}
\makeatletter
 % allow citations to break across lines
 \let\@cite@ofmt\@firstofone
 % avoid brackets around text for \cite:
 \def\@biblabel#1{}
 \def\@cite#1#2{{#1\if@tempswa , #2\fi}}
\makeatother
\newlength{\cslhangindent}
\setlength{\cslhangindent}{1.5em}
\newlength{\csllabelwidth}
\setlength{\csllabelwidth}{3em}
\newenvironment{CSLReferences}[2] % #1 hanging-indent, #2 entry-spacing
 {\begin{list}{}{%
  \setlength{\itemindent}{0pt}
  \setlength{\leftmargin}{0pt}
  \setlength{\parsep}{0pt}
  % turn on hanging indent if param 1 is 1
  \ifodd #1
   \setlength{\leftmargin}{\cslhangindent}
   \setlength{\itemindent}{-1\cslhangindent}
  \fi
  % set entry spacing
  \setlength{\itemsep}{#2\baselineskip}}}
 {\end{list}}
\usepackage{calc}
\newcommand{\CSLBlock}[1]{\hfill\break\parbox[t]{\linewidth}{\strut\ignorespaces#1\strut}}
\newcommand{\CSLLeftMargin}[1]{\parbox[t]{\csllabelwidth}{\strut#1\strut}}
\newcommand{\CSLRightInline}[1]{\parbox[t]{\linewidth - \csllabelwidth}{\strut#1\strut}}
\newcommand{\CSLIndent}[1]{\hspace{\cslhangindent}#1}

\usepackage{lineno}\linenumbers
\usepackage[noblocks]{authblk}
\renewcommand*{\Authsep}{, }
\renewcommand*{\Authand}{, }
\renewcommand*{\Authands}{, }
\renewcommand\Affilfont{\small}
\makeatletter
\@ifpackageloaded{caption}{}{\usepackage{caption}}
\AtBeginDocument{%
\ifdefined\contentsname
  \renewcommand*\contentsname{Table of contents}
\else
  \newcommand\contentsname{Table of contents}
\fi
\ifdefined\listfigurename
  \renewcommand*\listfigurename{List of Figures}
\else
  \newcommand\listfigurename{List of Figures}
\fi
\ifdefined\listtablename
  \renewcommand*\listtablename{List of Tables}
\else
  \newcommand\listtablename{List of Tables}
\fi
\ifdefined\figurename
  \renewcommand*\figurename{Figure}
\else
  \newcommand\figurename{Figure}
\fi
\ifdefined\tablename
  \renewcommand*\tablename{Table}
\else
  \newcommand\tablename{Table}
\fi
}
\@ifpackageloaded{float}{}{\usepackage{float}}
\floatstyle{ruled}
\@ifundefined{c@chapter}{\newfloat{codelisting}{h}{lop}}{\newfloat{codelisting}{h}{lop}[chapter]}
\floatname{codelisting}{Listing}
\newcommand*\listoflistings{\listof{codelisting}{List of Listings}}
\makeatother
\makeatletter
\makeatother
\makeatletter
\@ifpackageloaded{caption}{}{\usepackage{caption}}
\@ifpackageloaded{subcaption}{}{\usepackage{subcaption}}
\makeatother
\ifLuaTeX
  \usepackage{selnolig}  % disable illegal ligatures
\fi
\usepackage{bookmark}

\IfFileExists{xurl.sty}{\usepackage{xurl}}{} % add URL line breaks if available
\urlstyle{same} % disable monospaced font for URLs
\hypersetup{
  pdftitle={Computational methods in landscape ecology},
  pdfauthor={Maximilian H.K. Hesselbarth; Jakub Nowosad; Marti Bosch; Martin Jung; Rafael Schouten},
  pdfkeywords={keyword1, keyword2},
  colorlinks=true,
  linkcolor={blue},
  filecolor={Maroon},
  citecolor={Blue},
  urlcolor={Blue},
  pdfcreator={LaTeX via pandoc}}

\title{Computational methods in landscape ecology}


  \author{Maximilian H.K. Hesselbarth}
            \affil{%
                  International Institute for Applied Systems Analysis,
                  Biodiversity, Ecology, and Conservation Group,
                  Laxenburg, Austria
              }
        \author{Jakub Nowosad}
            \affil{%
                  Adam Mickiewicz University, Poznan, Poland
              }
        \author{Marti Bosch}
            \affil{%
                  École polytechnique fédérale de Lausanne, Lausanne,
                  Switzerland
              }
        \author{Martin Jung}
            \affil{%
                  International Institute for Applied Systems Analysis,
                  Biodiversity, Ecology, and Conservation Group,
                  Laxenburg, Austria
              }
        \author{Rafael Schouten}
            \affil{%
                  Globe Institute, University of Copenhagen, Copenhagen,
                  Denmark
              }
      
\date{}
\begin{document}
\maketitle

\setstretch{1.5}
\section{Introduction}\label{introduction}

Landscapes are typically defined as mosaics of different land covers,
habitats, ecosystems, or land use systems (Forman and Godron 1986;
Forman 1995) - with empathize on heterogeneity of at least one factor of
interest (Turner and Gardner 2015). Linking spatial heterogeneity and
ecological processes, including potential feedbacks, is the fundamental
concept of landscape ecology (Risser et al. 1984; Turner 1989, 2005).
Typical research topics include ecological flows in landscape mosaics,
drivers and consequences of land use and land cover (LULC) change,
nonlinear dynamics and landscape complexity, scale and scaling, spatial
analysis and landscape modeling, linking landscape metrics to ecological
processes, integrating humans and their activities into landscape
ecology, landscape conservation and sustainability, or accuracy
assessment and uncertainty analysis (Wu and Hobbs 2002; Wu 2013).

Computational science involves analyzing abstracted core mechanisms of
research questions using data and algorithms. Thus, computational
ecology can be defined as computational science that is used to address
ecological research questions with focus on complex, adaptive systems
and data-driven approaches (Poisot et al. 2019). Computational ecology
in general and simulations specifically are of importance for ecology
because ecological data is often context- and scale-dependent and
unfeasible to study in controllable, reproducible, and replicable
experimental settings (Petrovskii and Petrovskaya 2012; but see Wiersma
2022). Because landscape ecology is a cross-disciplinary field including
(e.g., social sciences, geography, or ecology and evolution; Wiens 1997;
With 2019), the general increase of data availability (Chi et al. 2016;
Jarić et al. 2020; Wüest et al. 2020; Nathan et al. 2022), or the
studied complex landscape systems (Newman et al. 2019), there is the
need for sophisticated computational methods in order to link
heterogeneity patterns and ecological processes. Thus, computational
methods are one of the most important tools of modern scientific
research (Prlić and Procter 2012; Wilson et al. 2014).

For all computational methods the natural world must be translated into
representative data models. The spatial vector data model is based on
points, lines, and polygons features that are defined by their geometry
and additional attributes related to them. Contrastingly, the raster
data model is based on evenly-spaced grid cells and discrete or
continuous values related to each cell (Lovelace et al. 2019; With
2019). The choice of the data models, spatial and temporal scales, or
the thematic resolution are crucial for the analysis (Wade et al. 2003;
Lechner and Rhodes 2016; Nedd et al. 2021).

Here, we aim to introduce the latest developments of computational
methods in landscape ecology. However, we do not provide a systematic
literature review, or a general introduction to (computational)
landscape ecology. For a more comprehensive, general introductions to
landscape ecology, please see Turner and Gardner (2015), Gergel and
Turner (2017), or With (2019). First, we will introduce recent
computational developments of the most important topics of landscape
ecology. This includes, for example, spatial patterns quantification,
connectivity and movement, or model simulations. Second, we will
introduce common software tools that implement or are potentially
capable of implementing these developments. However, due to its many
advantages (Powers and Hampton 2019), we will focus on open-source
software , and specifically scripting languages, i.e., R, Python, and
Julia. Last, we will discuss potential future needs developments of
computational methods in landscape ecology.

\section{Spatial patterns}\label{spatial-patterns}

Spatial patterns can be defined as the scale-dependent predictability of
the physical arrangement of observations (Dale 1999), or as clearly
identifiable structures in nature itself or data extracted from nature
(Grimm et al. 1996). Importantly, patterns observed in nature contain
information about the history of the system, such as demographic
processes, dispersal characteristics, or climatic patterns (Wiegand et
al. 2003). However, spatial patterns are not only a result of processes,
but could also be a driver of them. Thus, untangling the history of a
landscape and linking its spatial patterns to ecological processes is
one of the core concepts of landscape ecology (Turner 1989).

\subsection{Landscape metrics}\label{landscape-metrics}

Traditionally and still the most prominent approaches to quantify
spatial patterns revolve around raster data using categorical categories
based on the patch mosaic model (Frazier and Kedron 2017; Costanza et
al. 2019). The strength of landscape metrics is that they are comparable
easy to apply and communicate (Lausch et al. 2015), however, they are
also critiqued for not being able to realistically represent continuous
features of the natural word (Frazier and Kedron 2017). Metrics can be
calculated on patch-level (describing each patch), class-level
(describing all patches of the same class), or landscape level
(describing the entire landscape) and are able to quantify
characteristics related to e.g., patch areas and edge lengths, patch
shapes, or class aggregation and diversity (Hesselbarth et al. 2019).
Recently also several metrics describing landscape complexity were
introduced, e.g., several entropy measures (see following sections).

\subsection{Surface metrics}\label{surface-metrics}

Surface metrics are based on the gradient surface model using raster
data and continuous values (McGarigal et al. 2009; Cushman et al. 2010)
and were adapted from microscopy and molecular physics (McGarigal et al.
2009; Kedron et al. 2018). The gradient surface model can increase the
resemblance of the data model and the natural world because it allows to
include more heterogeneity within each grid cell (Cushman et al. 2010;
Kedron et al. 2018). Surface metrics are able to quantify various
characteristics, such as roughness, skewness and kurtosis, total and
relative amplitudes, curvatures of local peaks, or surface bearing area
ratios (Cushman et al. 2010). Many surface metrics have analogous
landscape metrics (Cushman et al. 2010), however, using surface metrics
allows to reveal different or additional patterns and potentially
pattern-process links (McGarigal et al. 2009; Borthwick et al. 2020).
Yet, software to calculate these metrics is still rare and further
research is needed into the specific pattern-process links and
ecological meaningful interpretations (Kedron et al. 2018; Frazier
2019).

\subsection{Entropy}\label{entropy}

It is mainly derived from information theory and thermodynamics, such as
Shannon entropy (and its analogous forms), Boltzmann entropy, or Renyi
entropy. Entropy in landscape ecology is mainly used to quantify the
complexity of the landscape (spatial heterogeneity), and less often
unpredictability (temporal heterogeneity) and scale dependence
(spatio-temporal heterogeneity; Vranken et al. 2015). However, studies
show that insights do not only depend on the formulation of the used
entropy, but also on the underlining data model, for example the
composition of categories or the co-occurrence matrix (Zhao and Zhang
2019).

The Shannon diversity index (Shannon 1948) quantifies the richness and
evenness of categories in the landscape omitting spatial configuration.
The index is based on information theory and is an adaptation of the
Shannon entropy which measures uncertainty of a random variable.
Shannon's entropy formula can also be modified to include the
landscape's spatial configuration, e.g., by weights calculated from
intra- and inter-class distances (Claramunt 2012). However, to quantify
the spatial configuration and the total complexity of the landscape,
other measures from information theory can be adapted. Nowosad and
Stepinski (2019) proposed to treat categorical raster as a bivariate
random variable \((x,y)\) where \(x\) is the category of the focal cell
and \(y\) is the category of the neighboring cell. This approach allows
us to compress information about the landscape's composition and
configuration into a co-occurrence matrix, which could be used to
calculate various entropy measures. This includes conditional entropy
(representing configurational complexity), joint entropy (representing
overall spatial-thematic complexity), mutual information, and relative
mutual information (representing the degree of spatial autocorrelation).

Based on notions from thermodynamics initiated by the work of Cushman
(2015) or Vranken et al. (2015), there was a surge in the development of
entropy-based metrics for landscape ecology in the last few years. In
the late 1800s, Boltzmann formulated a probabilistic interpretation of
the second law of thermodynamics using the concepts of macrostate (the
general state of a system) and microstate (the configuration of the
system elements; Gao and Li 2019). Cushman (2016) proposed to relate the
edge length to the microstate of the landscape, and use the proportion
of microstates to compute the relative Boltzmann entropy of a landscape
mosaic. This approach was generalized for calculations based on raster's
surface model and point patterns (Cushman 2021). Subsequently, Gao et
al. (2017) proposed to use the Boltzmann entropy to quantify the
complexity of a landscape surface by transforming the input raster into
a series of landscape surfaces with different levels of detail
(microstate) and calculating the Boltzmann entropy based on the total
number of microstates that are able to generate the observed macrostate.
Recently, Zhang et al. (2020) extended the original definition of
Boltzmann entropy to incorporate information about the adjacency of the
same categories in the landscape mosaic by using the number of
contiguous patches of the same category.

Moreover, many other entropy-based metrics have been proposed for use in
landscape ecology. The Renyi (Rényi 1961) and Gibbs entropies , which
are both generalizations of the Shannon entropy, are applied to quantify
the complexity of the landscape. The Rao quadratic entropy (Rao 1982)
has also been applied recently (Rocchini et al. 2017), as it measures
not only the relative abundances of elements, but also the pairwise
dissimilarities or distances between them. Thus, it can be useful in
cases, when the dissimilarities between LULC types, and not only their
categories, are relevant. Another recent development is the use of
Kullback-Leibler divergence (also known as relative entropy), which is a
measure of differences between two probability distributions to describe
patterns across scales (Huckeba et al. 2023).

\subsection{Landscape mosaic}\label{landscape-mosaic}

\subsection{Vector metrics}\label{vector-metrics}

Many landscape ecology studies are based on the raster data model.
However, the decision on the data model is often driven by data and
software availability, or the familiarity with the approach (Costanza et
al. 2019). For example, several early landscape ecology studies were
based on polygonal vector data that was converted to raster data. Yet,
nowadays, vector data is increasingly available, and most geospatial
software supports it.

Hundreds of metrics have already been developed for raster data, so the
most straightforward approach is to reimplement the same metrics for
vector data. This is possible for most of the metrics, with only several
exceptions of metrics that are specifically related to the raster data
model (Yao et al. 2022). Yet, the vector data model has been widely used
in other fields, such as geography or urban studies. For example, in
urban planning vector metrics are applied to quantify the shapes of
urban areas and characterize the complexity of building footprints
(Basaraner and Cetinkaya 2017). Earlier approaches included four
categories of compactness measures for vector data, namely
perimeter-area, single parameters of related circles, dispersion of
elements of the area around a centroid, and direct comparison to
standard shapes (Maceachren 1985). More recently, a unified theoretical
foundation for measuring shape compactness was introduced using a set of
ten distinct properties of a circle and metrics associated with each of
these properties (Angel et al. 2010)

Vector-based metrics also have their limitations. The most important is
computational complexity, which makes calculations of vector metrics
slower than their raster equivalents. Another technical issue is the
need for the data to be topologically correct, which is often
problematic especially for data from different sources. Last, the
pattern-process link is still less explored for vector-based metrics
compared to raster-based metrics.

\subsection{Operations on spatial
patterns}\label{operations-on-spatial-patterns}

Spatial patterns can be analyzed through a range of computational
operations, such as comparing, searching, or grouping (Turner 1990;
Wickham and Norton 1994; Jasiewicz et al. 2015; Remmel 2020; Nowosad
2021). These operations are based on spatial signatures which are
multi-numerical representations of landscape pattern, and dissimilarity
measures which are functions that quantify differences between the
signatures.

Spatial patterns can be compared between different areas, or for the
same area at two different moments in time. Comparing spatial patterns
is often used to analyze landscape dynamics, e.g., to detect changes in
landscape structure over time (Duncan and Boruff 2023). Furthermore,
pattern-based search is a one-to-many operation in which the spatial
signature of a query area is compared to the signatures of all other
areas. This allows to find areas with similar patterns compared to the
query area, e.g., areas with similar environmental conditions (Jasiewicz
et al. 2014). Finally, the grouping of spatial patterns is a
many-to-many operation in which spatial signatures are calculated for
all areas and clustered based on their similarity (Nowosad and Stepinski
2021).

\section{Simulation models}\label{simulation-models}

Simulation models are a powerful tool to study complex adaptive
ecological systems in controllable, reproducible, and replicable
settings (Pascual 2005; Petrovskii and Petrovskaya 2012). Thus, they can
be seen as experimental systems that allow, unlike to the natural world,
all imaginable manipulations in order to advance theoretical
developments or test hypotheses (Peck 2004). Due to the usual temporal
and spatial scales, complex interactions and feedbacks, or scale
mismatches between patterns and processes, simulation models are one of
the major approaches in landscape ecology (Schröder and Seppelt 2006;
Synes et al. 2016). In general, simulations models can be classified
using two major divisions, namely as \emph{i)} predictive or exploratory
models, and \emph{ii)} pattern- or process-based models (Peck 2004;
Synes et al. 2016). Here, we are going to focus mainly on exploratory
models, but include both pattern- and process-based models.

\subsection{Neutral landscapes models}\label{neutral-landscapes-models}

Neutral landscapes models simulate landscape patterns without assuming
any underlying abiotic or biotic processes. Thus, they are often used as
null hypothesis, baselines, or scenarios for comparisons with real
landscape patterns, while controlling certain aspects of the landscape
(Li et al. 2004; Wang and Malanson 2008). The first neutral models
simply assigned land cover types randomly to cells in the landscape and
were based on percolation theory (Gardner et al. 1987). Further
developments included hierarchical models that considered different
spatial scales while assigning land cover types (O'Neill et al. 1992).
While these first two classes can only simulate landscape with discrete
land cover classes, fractal neutral models can also simulate continuous
land cover maps by utilizing e.g., Brownian motion (Palmer 1992).
Borrowing from computer graphics, more recent neutral landscape models
make use of spectral synthesis (e.g., Perlin noise; Etherington 2022) or
Binary space partitioning (Etherington et al. 2022) hierarchical neutral
landscapes. In order to simulate landscapes dominated by human
activities approaches based on least-cost network movement exist
(Etherington et al. 2024). Neutral landscapes models can also be used to
simulated landscapes similar to real landscapes (Inkoom et al. 2017), or
based on target values of landscape metrics (Van Strien et al. 2016;
Justeau-Allaire et al. 2022).

\subsection{Landscape Generator}\label{landscape-generator}

Processes-based landscape generators are related to neutral landscapes
models because they can simulate virtual landscapes. However, in
contrast to neutral landscape models, processes-based landscape
generators are able to simulate structurally more realistic landscapes
by including pattern creating processes (Langhammer et al. 2019;
Salecker et al. 2019).

\subsection{IBMs /Coupled lattice}\label{ibms-coupled-lattice}

\subsection{Digital twins}\label{digital-twins}

\newpage{}

\subsubsection{References}\label{references}

\phantomsection\label{refs}
\begin{CSLReferences}{1}{1}
\bibitem[\citeproctext]{ref-Angel2010}
Angel S, Parent J, Civco DL (2010) Ten compactness properties of
circles: Measuring shape in geography. Canadian Geographies /
G{é}ographies canadiennes 54:441--461.
\url{https://doi.org/10.1111/j.1541-0064.2009.00304.x}

\bibitem[\citeproctext]{ref-Basaraner2017}
Basaraner M, Cetinkaya S (2017) Performance of shape indices and
classification schemes for characterising perceptual shape complexity of
building footprints in GIS. International Journal of Geographical
Information Science 31:1952--1977.
\url{https://doi.org/10.1080/13658816.2017.1346257}

\bibitem[\citeproctext]{ref-Borthwick2020}
Borthwick R, de Flamingh A, Hesselbarth MHK, et al (2020) Alternative
quantifications of landscape complementation to model gene flow in
banded longhorn beetles {[}typocerus v. Velutinus (olivier){]}.
Frontiers in Genetics 11:307.
\url{https://doi.org/10.3389/fgene.2020.00307}

\bibitem[\citeproctext]{ref-Chi2016}
Chi M, Plaza A, Benediktsson JA, et al (2016) Big data for remote
sensing: Challenges and opportunities. Proceedings of the IEEE
104:2207--2219. \url{https://doi.org/10.1109/JPROC.2016.2598228}

\bibitem[\citeproctext]{ref-Claramunt2012}
Claramunt C (2012)
\href{https://doi.org/10.1007/978-3-642-33999-8_28}{Towards a
spatio-temporal form of entropy}. In: Advances in conceptual modeling.
Springer Berlin Heidelberg, Berlin, Heidelberg, pp 221--230

\bibitem[\citeproctext]{ref-Costanza2019}
Costanza JK, Riitters K, Vogt P, Wickham J (2019) Describing and
analyzing landscape patterns: Where are we now, and where are we going?
Landscape Ecology 34:2049--2055.
\url{https://doi.org/10.1007/s10980-019-00889-6}

\bibitem[\citeproctext]{ref-Cushman2015}
Cushman SA (2015) Thermodynamics in landscape ecology: The importance of
integrating measurement and modeling of landscape entropy. Landscape
Ecology 30:7--10. \url{https://doi.org/10.1007/s10980-014-0108-x}

\bibitem[\citeproctext]{ref-Cushman2016}
Cushman SA (2016) Calculating the configurational entropy of a landscape
mosaic. Landscape Ecology 31:481--489.
\url{https://doi.org/10.1007/s10980-015-0305-2}

\bibitem[\citeproctext]{ref-Cushman2021a}
Cushman SA (2021) Generalizing boltzmann configurational entropy to
surfaces, point patterns and landscape mosaics. Entropy 23:1616.
\url{https://doi.org/10.3390/e23121616}

\bibitem[\citeproctext]{ref-Cushman2010}
Cushman SA, Gutzweiler K, Evans JS, McGarigal K (2010) The gradient
paradigm: A conceptual and analytical framework for landscape ecology.
In: Cushman SA, Huettmann F (eds) Spatial complexity informatics and
wildlife conservation. Springer International Publishing, Basel, CH, pp
83--108

\bibitem[\citeproctext]{ref-Dale1999}
Dale MRT (1999) \href{https://doi.org/10.1017/CBO9780511612589}{Spatial
pattern analysis in plant ecology}, 1st edn. Cambridge University Press

\bibitem[\citeproctext]{ref-Duncan2023}
Duncan JMA, Boruff B (2023) Monitoring spatial patterns of urban
vegetation: A comparison of contemporary high-resolution datasets.
Landscape and Urban Planning 233:104671.
\url{https://doi.org/10.1016/j.landurbplan.2022.104671}

\bibitem[\citeproctext]{ref-Etherington2022}
Etherington TR (2022) Perlin noise as a hierarchical neutral landscape
model. Web Ecology 22:1--6. \url{https://doi.org/10.5194/we-22-1-2022}

\bibitem[\citeproctext]{ref-Etherington2022a}
Etherington TR, Morgan FJ, O'Sullivan D (2022) Binary space partitioning
generates hierarchical and rectilinear neutral landscape models suitable
for human-dominated landscapes. Landscape Ecology 37:1761--1769.
\url{https://doi.org/10.1007/s10980-022-01452-6}

\bibitem[\citeproctext]{ref-Etherington2024}
Etherington TR, O'Sullivan D, Perry GLW, et al (2024) A least-cost
network neutral landscape model of human sites and routes. Landscape
Ecology 39:52. \url{https://doi.org/10.1007/s10980-024-01836-w}

\bibitem[\citeproctext]{ref-Forman1995}
Forman RTT (1995) Land mosaics: The ecology of landscapes and regions.
Cambridge University Press, Cambridge, UK

\bibitem[\citeproctext]{ref-Forman1986}
Forman RTT, Godron M (1986) Landscape ecology. Wiley \& Sons,
Chichester, UK

\bibitem[\citeproctext]{ref-Frazier2019}
Frazier AE (2019) Emerging trajectories for spatial pattern analysis in
landscape ecology. Landscape Ecology 34:2073--2082.
\url{https://doi.org/10.1007/s10980-019-00880-1}

\bibitem[\citeproctext]{ref-Frazier2017}
Frazier AE, Kedron P (2017) Landscape metrics: Past progress and future
directions. Current Landscape Ecology Reports 63--72.
\url{https://doi.org/10.1007/s40823-017-0026-0}

\bibitem[\citeproctext]{ref-Gao2019}
Gao P, Li Z (2019) Computation of the boltzmann entropy of a landscape:
A review and a generalization. Landscape Ecology 34:2183--2196.
\url{https://doi.org/10.1007/s10980-019-00814-x}

\bibitem[\citeproctext]{ref-Gao2017}
Gao P, Zhang H, Li Z (2017) A hierarchy-based solution to calculate the
configurational entropy of landscape gradients. Landscape Ecology.
\url{https://doi.org/10.1007/s10980-017-0515-x}

\bibitem[\citeproctext]{ref-Gardner1987}
Gardner RH, Milne BT, Turnei MG, O'Neill RV (1987) Neutral models for
the analysis of broad-scale landscape pattern. Landscape Ecology
1:19--28. \url{https://doi.org/10.1007/BF02275262}

\bibitem[\citeproctext]{ref-Gergel2017}
Gergel SE, Turner MG (eds) (2017)
\href{https://doi.org/10.1007/978-1-4939-6374-4}{Learning landscape
ecology}. Springer New York, New York, NY

\bibitem[\citeproctext]{ref-Grimm1996}
Grimm V, Frank K, Jeltsch F, et al (1996) Pattern-oriented modelling in
population ecology. The Science of the Total Environment 183:151--166.
\url{https://doi.org/10.1016/0048-9697(95)04966-5}

\bibitem[\citeproctext]{ref-Hesselbarth2019}
Hesselbarth MHK, Sciaini M, With KA, et al (2019) Landscapemetrics: An
open-source r tool to calculate landscape metrics. Ecography
42:1648--1657. \url{https://doi.org/10.1111/ecog.04617}

\bibitem[\citeproctext]{ref-Huckeba2023}
Huckeba G, Andresen B, Roach TNF (2023) Multi-scale spatial ecology
analyses: A kullback information approach. Landscape Ecology
38:645--657. \url{https://doi.org/10.1007/s10980-022-01514-9}

\bibitem[\citeproctext]{ref-Inkoom2017}
Inkoom JN, Frank S, Greve K, Fürst C (2017) Designing neutral landscapes
for data scarce regions in west africa. Ecological Informatics 42:1--13.
\url{https://doi.org/10.1016/j.ecoinf.2017.08.003}

\bibitem[\citeproctext]{ref-Jaric2020}
Jarić I, Correia RA, Brook BW, et al (2020) iEcology: Harnessing large
online resources to generate ecological insights. Trends in Ecology \&
Evolution 35:630--639. \url{https://doi.org/10.1016/j.tree.2020.03.003}

\bibitem[\citeproctext]{ref-Jasiewicz2015}
Jasiewicz J, Netzel P, Stepinski T (2015) GeoPAT: A toolbox for
pattern-based information retrieval from large geospatial databases.
Computers \& Geosciences 80:62--73.
\url{https://doi.org/10.1016/j.cageo.2015.04.002}

\bibitem[\citeproctext]{ref-Jasiewicz2014}
Jasiewicz J, Netzel P, Stepinski TF (2014) Landscape similarity,
retrieval, and machine mapping of physiographic units. Geomorphology
221:104--112. \url{https://doi.org/10.1016/j.geomorph.2014.06.011}

\bibitem[\citeproctext]{ref-Justeau-Allaire2022}
Justeau-Allaire D, Blanchard G, Ibanez T, et al (2022) Fragmented
landscape generator (flsgen): A neutral landscape generator with control
of landscape structure and fragmentation indices. Methods in Ecology and
Evolution 13:1412--1420. \url{https://doi.org/10.1111/2041-210X.13859}

\bibitem[\citeproctext]{ref-Kedron2018}
Kedron PJ, Frazier AE, Ovando-Montejo GA, Wang J (2018) Surface metrics
for landscape ecology: A comparison of landscape models across
ecoregions and scales. Landscape Ecology 33:1489--1504.
\url{https://doi.org/10.1007/s10980-018-0685-1}

\bibitem[\citeproctext]{ref-Langhammer2019}
Langhammer M, Thober J, Lange M, et al (2019) Agricultural landscape
generators for simulation models: A review of existing solutions and an
outline of future directions. Ecological Modelling 393:135--151.
\url{https://doi.org/10.1016/j.ecolmodel.2018.12.010}

\bibitem[\citeproctext]{ref-Lausch2015}
Lausch A, Blaschke T, Haase D, et al (2015) Understanding and
quantifying landscape structure - a review on relevant process
characteristics, data models and landscape metrics. Ecological Modelling
295:31--41. \url{https://doi.org/10.1016/j.ecolmodel.2014.08.018}

\bibitem[\citeproctext]{ref-Lechner2016}
Lechner AM, Rhodes JR (2016) Recent progress on spatial and thematic
resolution in landscape ecology. Current Landscape Ecology Reports
1:98--105. \url{https://doi.org/10.1007/s40823-016-0011-z}

\bibitem[\citeproctext]{ref-Li2004}
Li X, He HS, Wang X, et al (2004) Evaluating the effectiveness of
neutral landscape models to represent a real landscape. Landscape and
Urban Planning 69:137--148.
\url{https://doi.org/10.1016/j.landurbplan.2003.10.037}

\bibitem[\citeproctext]{ref-Lovelace2019}
Lovelace R, Nowosad J, Münchow J (2019) Geocomputation with r, 1st edn.
Chapman \& Hall/CRC, Boca Raton, USA

\bibitem[\citeproctext]{ref-Maceachren1985}
Maceachren AM (1985) Compactness of geographic shape: Comparison and
evaluation of measures. Geografiska Annaler: Series B, Human Geography
67:53--67. \url{https://doi.org/10.1080/04353684.1985.11879515}

\bibitem[\citeproctext]{ref-McGarigal2009}
McGarigal K, Tagil S, Cushman SA (2009) Surface metrics: An alternative
to patch metrics for the quantification of landscape structure.
Landscape Ecology 24:433--450.
\url{https://doi.org/10.1007/s10980-009-9327-y}

\bibitem[\citeproctext]{ref-Nathan2022}
Nathan R, Monk CT, Arlinghaus R, et al (2022) Big-data approaches lead
to an increased understanding of the ecology of animal movement. Science
375:eabg1780. \url{https://doi.org/10.1126/science.abg1780}

\bibitem[\citeproctext]{ref-Nedd2021}
Nedd R, Light K, Owens M, et al (2021) A synthesis of land use/land
cover studies: Definitions, classification systems, meta-studies,
challenges and knowledge gaps on a global landscape. Land 10:994.
\url{https://doi.org/10.3390/land10090994}

\bibitem[\citeproctext]{ref-Newman2019}
Newman EA, Kennedy MC, Falk DA, McKenzie D (2019) Scaling and complexity
in landscape ecology. Frontiers in Ecology and Evolution 7:293.
\url{https://doi.org/10.3389/fevo.2019.00293}

\bibitem[\citeproctext]{ref-Nowosad2021a}
Nowosad J (2021) Motif: An open-source r tool for pattern-based spatial
analysis. Landscape Ecology 36:29--43.
\url{https://doi.org/10.1007/s10980-020-01135-0}

\bibitem[\citeproctext]{ref-Nowosad2019a}
Nowosad J, Stepinski TF (2019) Information theory as a consistent
framework for quantification and classification of landscape patterns.
Landscape Ecology 34:2091--2101.
\url{https://doi.org/10.1007/s10980-019-00830-x}

\bibitem[\citeproctext]{ref-Nowosad2021b}
Nowosad J, Stepinski TF (2021) Pattern-based identification and mapping
of landscape types using multi-thematic data. International Journal of
Geographical Information Science 35:1634--1649.
\url{https://doi.org/10.1080/13658816.2021.1893324}

\bibitem[\citeproctext]{ref-ONeill1992}
O'Neill RV, Gardner RH, Turner MG (1992) A hierarchical neutral model
for landscape analysis. Landscape Ecology 7:55--61.
\url{https://doi.org/10.1007/BF02573957}

\bibitem[\citeproctext]{ref-Palmer1992}
Palmer MW (1992) The coexistence of species in fractal landscapes. The
American Naturalist 139:375--397. \url{https://doi.org/10.1086/285332}

\bibitem[\citeproctext]{ref-Pascual2005}
Pascual M (2005) Computational ecology: From the complex to the simple
and back. PLoS Computional Biology 1:e18.
\url{https://doi.org/10.1371/journal.pcbi.0010018}

\bibitem[\citeproctext]{ref-Peck2004}
Peck SL (2004) Simulation as experiment: A philosophical reassessment
for biological modeling. Trends in Ecology \& Evolution 19:530--534.
\url{https://doi.org/10.1016/j.tree.2004.07.019}

\bibitem[\citeproctext]{ref-Petrovskii2012}
Petrovskii S, Petrovskaya N (2012) Computational ecology as an emerging
science. Interface Focus 2:241--254.
\url{https://doi.org/10.1098/rsfs.2011.0083}

\bibitem[\citeproctext]{ref-Poisot2019}
Poisot T, LaBrie R, Larson E, et al (2019) Data-based, synthesis-driven:
Setting the agenda for computational ecology. Ideas in Ecology and
Evolution 12: \url{https://doi.org/10.24908/iee.2019.12.2.e}

\bibitem[\citeproctext]{ref-Powers2019}
Powers SM, Hampton SE (2019) Open science, reproducibility, and
transparency in ecology. Ecological Applications 29:e01822.
\url{https://doi.org/10.1002/eap.1822}

\bibitem[\citeproctext]{ref-Prlic2012}
Prlić A, Procter JB (2012) Ten simple rules for the open development of
scientific software. PLoS Computational Biology 8:e1002802.
\url{https://doi.org/10.1371/journal.pcbi.1002802}

\bibitem[\citeproctext]{ref-Rao1982}
Rao CR (1982) Diversity and dissimilarity coefficients: A unified
approach. Theoretical Population Biology 21:24--43.
\url{https://doi.org/10.1016/0040-5809(82)90004-1}

\bibitem[\citeproctext]{ref-Remmel2020}
Remmel TK (2020) Distributions of hyper-local configuration elements to
characterize, compare, and assess landscape-level spatial patterns.
Entropy 22:420. \url{https://doi.org/10.3390/e22040420}

\bibitem[\citeproctext]{ref-Renyi1961}
Rényi A (1961) On measures of entropy and information. In: Proceedings
of the fourth berkeley symposium on mathematical statistics and
probability, volume 1: Contributions to the theory of statistics.
University of California Press, pp 547--562

\bibitem[\citeproctext]{ref-Risser1984}
Risser PG, Karr JR, Forman RTT (1984) Landscape ecology: Directions and
approaches. Illinois Natural History Survery Special Publication 2:7--14

\bibitem[\citeproctext]{ref-Rocchini2017}
Rocchini D, Marcantonio M, Ricotta C (2017) Measuring rao's q diversity
index from remote sensing: An open source solution. Ecological
Indicators 72:234--238.
\url{https://doi.org/10.1016/j.ecolind.2016.07.039}

\bibitem[\citeproctext]{ref-Salecker2019}
Salecker J, Dislich C, Wiegand K, et al (2019) EFForTS-LGraf: A
landscape generator for creating smallholder-driven land-use mosaics.
PLOS ONE 14:e0222949. \url{https://doi.org/10.1371/journal.pone.0222949}

\bibitem[\citeproctext]{ref-Schroder2006}
Schröder B, Seppelt R (2006) Analysis of pattern--process interactions
based on landscape models - overview, general concepts, and
methodological issues. Ecological Modelling 199:505--516.
\url{https://doi.org/10.1016/j.ecolmodel.2006.05.036}

\bibitem[\citeproctext]{ref-Shannon1948}
Shannon CE (1948) A mathematical theory of communication. Bell System
Technical Journal 27:379--423.
\url{https://doi.org/10.1002/j.1538-7305.1948.tb01338.x}

\bibitem[\citeproctext]{ref-Synes2016}
Synes NW, Brown C, Watts K, et al (2016) Emerging opportunities for
landscape ecological modelling. Current Landscape Ecology Reports
1:146--167

\bibitem[\citeproctext]{ref-Turner1989}
Turner MG (1989) Landscape ecology: The effect of pattern on process.
Annual Review of Ecology and Systematics 20:171--197.
\url{https://doi.org/10.1146/annurev.es.20.110189.001131}

\bibitem[\citeproctext]{ref-Turner1990}
Turner MG (1990) Spatial and temporal analysis of landscape patterns.
Landscape Ecology 4:21--30. \url{https://doi.org/10.1007/BF02573948}

\bibitem[\citeproctext]{ref-Turner2005}
Turner MG (2005) Landscape ecology: What is the state of the science?
Annual Review of Ecology, Evolution, and Systematics 36:319--344.
\url{https://doi.org/10.1146/annurev.ecolsys.36.102003.152614}

\bibitem[\citeproctext]{ref-Turner2015}
Turner MG, Gardner RH (2015) Landscape ecology in theory and practice:
Pattern and process, 2nd edition. Springer, New York

\bibitem[\citeproctext]{ref-VanStrien2016}
Van Strien MJ, Slager CTJ, De Vries B, Grêt-Regamey A (2016) An improved
neutral landscape model for recreating real landscapes and generating
landscape series for spatial ecological simulations. Ecology and
Evolution 6:3808--3821. \url{https://doi.org/10.1002/ece3.2145}

\bibitem[\citeproctext]{ref-Vranken2015}
Vranken I, Baudry J, Aubinet M, et al (2015) A review on the use of
entropy in landscape ecology: Heterogeneity, unpredictability, scale
dependence and their links with thermodynamics. Landscape Ecology
30:51--65. \url{https://doi.org/10.1007/s10980-014-0105-0}

\bibitem[\citeproctext]{ref-Wade2003}
Wade TG, Wickham JD, Nash MS, et al (2003) A comparison of vector and
raster GIS methods for calculating landscape metrics used in
environmental assessments. Photogrammetric Engineering \& Remote Sensing
69:1399--1405. \url{https://doi.org/10.14358/PERS.69.12.1399}

\bibitem[\citeproctext]{ref-Wang2008}
Wang Q, Malanson GP (2008) Neutral landscapes: Bases for exploration in
landscape ecology. Geography Compass 2:319--339.
\url{https://doi.org/10.1111/j.1749-8198.2008.00090.x}

\bibitem[\citeproctext]{ref-Wickham1994}
Wickham JD, Norton DJ (1994) Mapping and analyzing landscape patterns.
Landscape Ecology 9:7--23. \url{https://doi.org/10.1007/BF00135075}

\bibitem[\citeproctext]{ref-Wiegand2003}
Wiegand T, Jeltsch F, Hanski I, Grimm V (2003) Using pattern-oriented
modeling for revealing hidden information: A key for reconciling
ecological theory and application. Oikos 100:209--222.
\url{https://doi.org/10.1034/j.1600-0706.2003.12027.x}

\bibitem[\citeproctext]{ref-Wiens1997}
Wiens JA (1997)
\href{https://doi.org/10.1016/B978-012323445-2/50005-5}{Metapopulation
dynamics and landscape ecology}. In: Metapopulation biology. Elsevier,
pp 43--62

\bibitem[\citeproctext]{ref-Wiersma2022}
Wiersma YF (2022) A review of landscape ecology experiments to
understand ecological processes. Ecological Processes 11:57.
\url{https://doi.org/10.1186/s13717-022-00401-0}

\bibitem[\citeproctext]{ref-Wilson2014}
Wilson G, Aruliah DA, Brown CT, et al (2014) Best practices for
scientific computing. PLoS Biology 12:e1001745.
\url{https://doi.org/10.1371/journal.pbio.1001745}

\bibitem[\citeproctext]{ref-With2019}
With KA (2019) Essentials of landscape ecology, 1st edn. Oxford
University Press, Oxford, UK

\bibitem[\citeproctext]{ref-Wu2013}
Wu J (2013) Key concepts and research topics in landscape ecology
revisited: 30 years after the allerton park workshop. Landscape Ecology
28:1--11. \url{https://doi.org/10.1007/s10980-012-9836-y}

\bibitem[\citeproctext]{ref-Wu2002}
Wu J, Hobbs R (2002) Key issues and research priorities in landscape
ecology: An idiosyncratic synthesis. Landscape Ecology 17:355--365.
\url{https://doi.org/10.1023/A:1020561630963}

\bibitem[\citeproctext]{ref-Wuest2020}
Wüest RO, Zimmermann NE, Zurell D, et al (2020) Macroecology in the age
of big data -- where to go from here? Journal of Biogeography 47:1--12.
\url{https://doi.org/10.1111/jbi.13633}

\bibitem[\citeproctext]{ref-Yao2022}
Yao Y, Cheng T, Sun Z, et al (2022) VecLI: A framework for calculating
vector landscape indices considering landscape fragmentation.
Environmental Modelling and Software 149:105325.
\url{https://doi.org/10.1016/j.envsoft.2022.105325}

\bibitem[\citeproctext]{ref-Zhang2020}
Zhang H, Wu Z, Lan T, et al (2020) Calculating the wasserstein
metric-based boltzmann entropy of a landscape mosaic. Entropy 22:381.
\url{https://doi.org/10.3390/e22040381}

\bibitem[\citeproctext]{ref-Zhao2019}
Zhao Y, Zhang X (2019) Calculating spatial configurational entropy of a
landscape mosaic based on the wasserstein metric. Landscape Ecology
34:1849--1858. \url{https://doi.org/10.1007/s10980-019-00876-x}

\end{CSLReferences}



\end{document}
