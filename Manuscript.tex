% Options for packages loaded elsewhere
\PassOptionsToPackage{unicode}{hyperref}
\PassOptionsToPackage{hyphens}{url}
\PassOptionsToPackage{dvipsnames,svgnames,x11names}{xcolor}
%
\documentclass[
  12pt,
  a4paperpaper,
]{article}

\usepackage{amsmath,amssymb}
\usepackage{setspace}
\usepackage{iftex}
\ifPDFTeX
  \usepackage[T1]{fontenc}
  \usepackage[utf8]{inputenc}
  \usepackage{textcomp} % provide euro and other symbols
\else % if luatex or xetex
  \usepackage{unicode-math}
  \defaultfontfeatures{Scale=MatchLowercase}
  \defaultfontfeatures[\rmfamily]{Ligatures=TeX,Scale=1}
\fi
\usepackage{lmodern}
\ifPDFTeX\else  
    % xetex/luatex font selection
\fi
% Use upquote if available, for straight quotes in verbatim environments
\IfFileExists{upquote.sty}{\usepackage{upquote}}{}
\IfFileExists{microtype.sty}{% use microtype if available
  \usepackage[]{microtype}
  \UseMicrotypeSet[protrusion]{basicmath} % disable protrusion for tt fonts
}{}
\usepackage{xcolor}
\usepackage[margin=25mm]{geometry}
\setlength{\emergencystretch}{3em} % prevent overfull lines
\setcounter{secnumdepth}{-\maxdimen} % remove section numbering
% Make \paragraph and \subparagraph free-standing
\ifx\paragraph\undefined\else
  \let\oldparagraph\paragraph
  \renewcommand{\paragraph}[1]{\oldparagraph{#1}\mbox{}}
\fi
\ifx\subparagraph\undefined\else
  \let\oldsubparagraph\subparagraph
  \renewcommand{\subparagraph}[1]{\oldsubparagraph{#1}\mbox{}}
\fi


\providecommand{\tightlist}{%
  \setlength{\itemsep}{0pt}\setlength{\parskip}{0pt}}\usepackage{longtable,booktabs,array}
\usepackage{calc} % for calculating minipage widths
% Correct order of tables after \paragraph or \subparagraph
\usepackage{etoolbox}
\makeatletter
\patchcmd\longtable{\par}{\if@noskipsec\mbox{}\fi\par}{}{}
\makeatother
% Allow footnotes in longtable head/foot
\IfFileExists{footnotehyper.sty}{\usepackage{footnotehyper}}{\usepackage{footnote}}
\makesavenoteenv{longtable}
\usepackage{graphicx}
\makeatletter
\def\maxwidth{\ifdim\Gin@nat@width>\linewidth\linewidth\else\Gin@nat@width\fi}
\def\maxheight{\ifdim\Gin@nat@height>\textheight\textheight\else\Gin@nat@height\fi}
\makeatother
% Scale images if necessary, so that they will not overflow the page
% margins by default, and it is still possible to overwrite the defaults
% using explicit options in \includegraphics[width, height, ...]{}
\setkeys{Gin}{width=\maxwidth,height=\maxheight,keepaspectratio}
% Set default figure placement to htbp
\makeatletter
\def\fps@figure{htbp}
\makeatother
% definitions for citeproc citations
\NewDocumentCommand\citeproctext{}{}
\NewDocumentCommand\citeproc{mm}{%
  \begingroup\def\citeproctext{#2}\cite{#1}\endgroup}
\makeatletter
 % allow citations to break across lines
 \let\@cite@ofmt\@firstofone
 % avoid brackets around text for \cite:
 \def\@biblabel#1{}
 \def\@cite#1#2{{#1\if@tempswa , #2\fi}}
\makeatother
\newlength{\cslhangindent}
\setlength{\cslhangindent}{1.5em}
\newlength{\csllabelwidth}
\setlength{\csllabelwidth}{3em}
\newenvironment{CSLReferences}[2] % #1 hanging-indent, #2 entry-spacing
 {\begin{list}{}{%
  \setlength{\itemindent}{0pt}
  \setlength{\leftmargin}{0pt}
  \setlength{\parsep}{0pt}
  % turn on hanging indent if param 1 is 1
  \ifodd #1
   \setlength{\leftmargin}{\cslhangindent}
   \setlength{\itemindent}{-1\cslhangindent}
  \fi
  % set entry spacing
  \setlength{\itemsep}{#2\baselineskip}}}
 {\end{list}}
\usepackage{calc}
\newcommand{\CSLBlock}[1]{\hfill\break\parbox[t]{\linewidth}{\strut\ignorespaces#1\strut}}
\newcommand{\CSLLeftMargin}[1]{\parbox[t]{\csllabelwidth}{\strut#1\strut}}
\newcommand{\CSLRightInline}[1]{\parbox[t]{\linewidth - \csllabelwidth}{\strut#1\strut}}
\newcommand{\CSLIndent}[1]{\hspace{\cslhangindent}#1}

\usepackage{lineno}\linenumbers
\usepackage[noblocks]{authblk}
\renewcommand*{\Authsep}{, }
\renewcommand*{\Authand}{, }
\renewcommand*{\Authands}{, }
\renewcommand\Affilfont{\small}
\makeatletter
\@ifpackageloaded{caption}{}{\usepackage{caption}}
\AtBeginDocument{%
\ifdefined\contentsname
  \renewcommand*\contentsname{Table of contents}
\else
  \newcommand\contentsname{Table of contents}
\fi
\ifdefined\listfigurename
  \renewcommand*\listfigurename{List of Figures}
\else
  \newcommand\listfigurename{List of Figures}
\fi
\ifdefined\listtablename
  \renewcommand*\listtablename{List of Tables}
\else
  \newcommand\listtablename{List of Tables}
\fi
\ifdefined\figurename
  \renewcommand*\figurename{Figure}
\else
  \newcommand\figurename{Figure}
\fi
\ifdefined\tablename
  \renewcommand*\tablename{Table}
\else
  \newcommand\tablename{Table}
\fi
}
\@ifpackageloaded{float}{}{\usepackage{float}}
\floatstyle{ruled}
\@ifundefined{c@chapter}{\newfloat{codelisting}{h}{lop}}{\newfloat{codelisting}{h}{lop}[chapter]}
\floatname{codelisting}{Listing}
\newcommand*\listoflistings{\listof{codelisting}{List of Listings}}
\makeatother
\makeatletter
\makeatother
\makeatletter
\@ifpackageloaded{caption}{}{\usepackage{caption}}
\@ifpackageloaded{subcaption}{}{\usepackage{subcaption}}
\makeatother
\ifLuaTeX
  \usepackage{selnolig}  % disable illegal ligatures
\fi
\usepackage{bookmark}

\IfFileExists{xurl.sty}{\usepackage{xurl}}{} % add URL line breaks if available
\urlstyle{same} % disable monospaced font for URLs
\hypersetup{
  pdftitle={Computational methods in landscape ecology},
  pdfauthor={Maximilian H.K. Hesselbarth},
  pdfkeywords={keyword1, keyword2},
  colorlinks=true,
  linkcolor={blue},
  filecolor={Maroon},
  citecolor={Blue},
  urlcolor={Blue},
  pdfcreator={LaTeX via pandoc}}

\title{Computational methods in landscape ecology}


  \author{Maximilian H.K. Hesselbarth}
            \affil{%
                  International Institute for Applied Systems Analysis,
                  Biodiversity, Ecology, and Conservation Group,
                  Laxenburg (Austria)
              }
      
\date{}
\begin{document}
\maketitle

\setstretch{1.75}
\section{Introduction}\label{introduction}

Landscapes are typically defined as mosaics of different land covers,
habitats, ecosystems, or land use systems (Forman and Godron 1986;
Forman 1995) - with empathize on existing heterogeneity in at least one
factor of interest (Turner and Gardner 2015). Linking the spatial
heterogeneity and abiotic as well as biotic processes is the fundamental
concept of landscape ecology. This includes potential feedbacks between
heterogeneity and processes, spatial and temporal flows across
heterogenous landscapes, or the management of heterogeneity in
landscapes (Risser et al. 1984; Turner 1989). However, due to the
spatial context of heterogeneity, the pattern-process link is scale
dependent (Wiens 1989; Levin 1992; Wu 2013).

Computational science involves analyzing abstracted core mechanisms of
research questions using data and algorithms. Thus, computational
ecology can be defined if computational science is used to address
ecological research questions with focus on complex adaptive systems and
data-driven approaches (Poisot et al. 2019). Computational ecology in
general and simulations specifically are important for ecology because
ecological data is often context- and scale-dependent and infeasible to
study in reproducible, replicable and controlled experimental settings
(Petrovskii and Petrovskaya 2012).

Because landscape ecology is a cross-disciplinary field including (e.g.,
social sciences, geography, or ecology and evolution Wiens 1997; With
2019), the general increase of available data (Chi et al. 2016; Jarić et
al. 2020; Wüest et al. 2020; Nathan et al. 2022), or the complex systems
characteristics of landscapes and processes (Newman et al. 2019), there
is the need for sophisticated computational methods in order to link
patterns and processes.

This paper aims to introduce the latest developments of computation
methods in landscape ecology. However, it is not a general introduction
into (computational) landscape ecology. For a more comprehensive
introductions to landscape ecology in general, please see e.g.~Turner
and Gardner (2015) or Gergel and Turner (2017).

\section{Spatial patterns}\label{spatial-patterns}

Landscape ecology is fun (Turner 1989). There are many things to explore
(With 2019).

\subsection{Landscape metrics}\label{landscape-metrics}

\subsection{Vector metrics}\label{vector-metrics}

\subsection{Surface metrics}\label{surface-metrics}

\subsection{Entropy}\label{entropy}

\subsection{Landscape Mosaic}\label{landscape-mosaic}

\section{Spatial planning}\label{spatial-planning}

Landscape ecology is fun (Turner 1989). There are many things to explore
(With 2019).

\newpage{}

\subsubsection{References}\label{references}

\phantomsection\label{refs}
\begin{CSLReferences}{1}{1}
\bibitem[\citeproctext]{ref-Chi2016}
Chi M, Plaza A, Benediktsson JA, et al (2016) Big data for remote
sensing: Challenges and opportunities. Proceedings of the IEEE
104:2207--2219. \url{https://doi.org/10.1109/JPROC.2016.2598228}

\bibitem[\citeproctext]{ref-Forman1995}
Forman RTT (1995) Land mosaics: The ecology of landscapes and regions.
Cambridge University Press, Cambridge, UK

\bibitem[\citeproctext]{ref-Forman1986}
Forman RTT, Godron M (1986) Landscape ecology. Wiley \& Sons,
Chichester, UK

\bibitem[\citeproctext]{ref-Gergel2017}
Gergel SE, Turner MG (eds) (2017)
\href{https://doi.org/10.1007/978-1-4939-6374-4}{Learning landscape
ecology}. Springer New York, New York, NY

\bibitem[\citeproctext]{ref-Jaric2020}
Jarić I, Correia RA, Brook BW, et al (2020) iEcology: Harnessing large
online resources to generate ecological insights. Trends in Ecology \&
Evolution 35:630--639. \url{https://doi.org/10.1016/j.tree.2020.03.003}

\bibitem[\citeproctext]{ref-Levin1992}
Levin SA (1992) The problem of pattern and scale in ecology. Ecology
73:1943--1967. \url{https://doi.org/10.2307/1941447}

\bibitem[\citeproctext]{ref-Nathan2022}
Nathan R, Monk CT, Arlinghaus R, et al (2022) Big-data approaches lead
to an increased understanding of the ecology of animal movement. Science
375:eabg1780. \url{https://doi.org/10.1126/science.abg1780}

\bibitem[\citeproctext]{ref-Newman2019}
Newman EA, Kennedy MC, Falk DA, McKenzie D (2019) Scaling and complexity
in landscape ecology. Frontiers in Ecology and Evolution 7:293.
\url{https://doi.org/10.3389/fevo.2019.00293}

\bibitem[\citeproctext]{ref-Petrovskii2012}
Petrovskii S, Petrovskaya N (2012) Computational ecology as an emerging
science. Interface Focus 2:241--254.
\url{https://doi.org/10.1098/rsfs.2011.0083}

\bibitem[\citeproctext]{ref-Poisot2019}
Poisot T, LaBrie R, Larson E, et al (2019) Data-based, synthesis-driven:
Setting the agenda for computational ecology. Ideas in Ecology and
Evolution 12: \url{https://doi.org/10.24908/iee.2019.12.2.e}

\bibitem[\citeproctext]{ref-Risser1984}
Risser PG, Karr JR, Forman RTT (1984) Landscape ecology: Directions and
approaches. Illinois Natural History Survery Special Publication 2:7--14

\bibitem[\citeproctext]{ref-Turner1989}
Turner MG (1989) Landscape ecology: The effect of pattern on process.
Annual Review of Ecology and Systematics 20:171--197.
\url{https://doi.org/10.1146/annurev.es.20.110189.001131}

\bibitem[\citeproctext]{ref-Turner2015}
Turner MG, Gardner RH (2015) Landscape ecology in theory and practice:
Pattern and process, 2nd edition. Springer, New York

\bibitem[\citeproctext]{ref-Wiens1989}
Wiens JA (1989) Spatial scaling in ecology. Functional Ecology
3:385--397. \url{https://doi.org/10.2307/2389612}

\bibitem[\citeproctext]{ref-Wiens1997}
Wiens JA (1997)
\href{https://doi.org/10.1016/B978-012323445-2/50005-5}{Metapopulation
dynamics and landscape ecology}. In: Metapopulation biology. Elsevier,
pp 43--62

\bibitem[\citeproctext]{ref-With2019}
With KA (2019) Essentials of landscape ecology, 1st edn. Oxford
University Press, Oxford, UK

\bibitem[\citeproctext]{ref-Wu2013}
Wu J (2013) Key concepts and research topics in landscape ecology
revisited: 30 years after the allerton park workshop. Landscape Ecology
28:1--11. \url{https://doi.org/10.1007/s10980-012-9836-y}

\bibitem[\citeproctext]{ref-Wuest2020}
Wüest RO, Zimmermann NE, Zurell D, et al (2020) Macroecology in the age
of big data -- where to go from here? Journal of Biogeography 47:1--12.
\url{https://doi.org/10.1111/jbi.13633}

\end{CSLReferences}



\end{document}
