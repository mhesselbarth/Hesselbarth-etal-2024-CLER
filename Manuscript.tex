% Options for packages loaded elsewhere
\PassOptionsToPackage{unicode}{hyperref}
\PassOptionsToPackage{hyphens}{url}
\PassOptionsToPackage{dvipsnames,svgnames,x11names}{xcolor}
%
\documentclass[
  12pt,
  a4paperpaper,
]{article}

\usepackage{amsmath,amssymb}
\usepackage{setspace}
\usepackage{iftex}
\ifPDFTeX
  \usepackage[T1]{fontenc}
  \usepackage[utf8]{inputenc}
  \usepackage{textcomp} % provide euro and other symbols
\else % if luatex or xetex
  \usepackage{unicode-math}
  \defaultfontfeatures{Scale=MatchLowercase}
  \defaultfontfeatures[\rmfamily]{Ligatures=TeX,Scale=1}
\fi
\usepackage{lmodern}
\ifPDFTeX\else  
    % xetex/luatex font selection
\fi
% Use upquote if available, for straight quotes in verbatim environments
\IfFileExists{upquote.sty}{\usepackage{upquote}}{}
\IfFileExists{microtype.sty}{% use microtype if available
  \usepackage[]{microtype}
  \UseMicrotypeSet[protrusion]{basicmath} % disable protrusion for tt fonts
}{}
\usepackage{xcolor}
\usepackage[margin=25mm]{geometry}
\setlength{\emergencystretch}{3em} % prevent overfull lines
\setcounter{secnumdepth}{-\maxdimen} % remove section numbering
% Make \paragraph and \subparagraph free-standing
\ifx\paragraph\undefined\else
  \let\oldparagraph\paragraph
  \renewcommand{\paragraph}[1]{\oldparagraph{#1}\mbox{}}
\fi
\ifx\subparagraph\undefined\else
  \let\oldsubparagraph\subparagraph
  \renewcommand{\subparagraph}[1]{\oldsubparagraph{#1}\mbox{}}
\fi


\providecommand{\tightlist}{%
  \setlength{\itemsep}{0pt}\setlength{\parskip}{0pt}}\usepackage{longtable,booktabs,array}
\usepackage{calc} % for calculating minipage widths
% Correct order of tables after \paragraph or \subparagraph
\usepackage{etoolbox}
\makeatletter
\patchcmd\longtable{\par}{\if@noskipsec\mbox{}\fi\par}{}{}
\makeatother
% Allow footnotes in longtable head/foot
\IfFileExists{footnotehyper.sty}{\usepackage{footnotehyper}}{\usepackage{footnote}}
\makesavenoteenv{longtable}
\usepackage{graphicx}
\makeatletter
\def\maxwidth{\ifdim\Gin@nat@width>\linewidth\linewidth\else\Gin@nat@width\fi}
\def\maxheight{\ifdim\Gin@nat@height>\textheight\textheight\else\Gin@nat@height\fi}
\makeatother
% Scale images if necessary, so that they will not overflow the page
% margins by default, and it is still possible to overwrite the defaults
% using explicit options in \includegraphics[width, height, ...]{}
\setkeys{Gin}{width=\maxwidth,height=\maxheight,keepaspectratio}
% Set default figure placement to htbp
\makeatletter
\def\fps@figure{htbp}
\makeatother
% definitions for citeproc citations
\NewDocumentCommand\citeproctext{}{}
\NewDocumentCommand\citeproc{mm}{%
  \begingroup\def\citeproctext{#2}\cite{#1}\endgroup}
\makeatletter
 % allow citations to break across lines
 \let\@cite@ofmt\@firstofone
 % avoid brackets around text for \cite:
 \def\@biblabel#1{}
 \def\@cite#1#2{{#1\if@tempswa , #2\fi}}
\makeatother
\newlength{\cslhangindent}
\setlength{\cslhangindent}{1.5em}
\newlength{\csllabelwidth}
\setlength{\csllabelwidth}{3em}
\newenvironment{CSLReferences}[2] % #1 hanging-indent, #2 entry-spacing
 {\begin{list}{}{%
  \setlength{\itemindent}{0pt}
  \setlength{\leftmargin}{0pt}
  \setlength{\parsep}{0pt}
  % turn on hanging indent if param 1 is 1
  \ifodd #1
   \setlength{\leftmargin}{\cslhangindent}
   \setlength{\itemindent}{-1\cslhangindent}
  \fi
  % set entry spacing
  \setlength{\itemsep}{#2\baselineskip}}}
 {\end{list}}
\usepackage{calc}
\newcommand{\CSLBlock}[1]{\hfill\break\parbox[t]{\linewidth}{\strut\ignorespaces#1\strut}}
\newcommand{\CSLLeftMargin}[1]{\parbox[t]{\csllabelwidth}{\strut#1\strut}}
\newcommand{\CSLRightInline}[1]{\parbox[t]{\linewidth - \csllabelwidth}{\strut#1\strut}}
\newcommand{\CSLIndent}[1]{\hspace{\cslhangindent}#1}

\usepackage{lineno}\linenumbers
\usepackage[noblocks]{authblk}
\renewcommand*{\Authsep}{, }
\renewcommand*{\Authand}{, }
\renewcommand*{\Authands}{, }
\renewcommand\Affilfont{\small}
\makeatletter
\@ifpackageloaded{caption}{}{\usepackage{caption}}
\AtBeginDocument{%
\ifdefined\contentsname
  \renewcommand*\contentsname{Table of contents}
\else
  \newcommand\contentsname{Table of contents}
\fi
\ifdefined\listfigurename
  \renewcommand*\listfigurename{List of Figures}
\else
  \newcommand\listfigurename{List of Figures}
\fi
\ifdefined\listtablename
  \renewcommand*\listtablename{List of Tables}
\else
  \newcommand\listtablename{List of Tables}
\fi
\ifdefined\figurename
  \renewcommand*\figurename{Figure}
\else
  \newcommand\figurename{Figure}
\fi
\ifdefined\tablename
  \renewcommand*\tablename{Table}
\else
  \newcommand\tablename{Table}
\fi
}
\@ifpackageloaded{float}{}{\usepackage{float}}
\floatstyle{ruled}
\@ifundefined{c@chapter}{\newfloat{codelisting}{h}{lop}}{\newfloat{codelisting}{h}{lop}[chapter]}
\floatname{codelisting}{Listing}
\newcommand*\listoflistings{\listof{codelisting}{List of Listings}}
\makeatother
\makeatletter
\makeatother
\makeatletter
\@ifpackageloaded{caption}{}{\usepackage{caption}}
\@ifpackageloaded{subcaption}{}{\usepackage{subcaption}}
\makeatother
\ifLuaTeX
  \usepackage{selnolig}  % disable illegal ligatures
\fi
\usepackage{bookmark}

\IfFileExists{xurl.sty}{\usepackage{xurl}}{} % add URL line breaks if available
\urlstyle{same} % disable monospaced font for URLs
\hypersetup{
  pdftitle={Computational methods in landscape ecology},
  pdfauthor={Maximilian H.K. Hesselbarth; Jakub Nowosad; Marti Bosch; Martin Jung; Rafael Schouten},
  pdfkeywords={keyword1, keyword2},
  colorlinks=true,
  linkcolor={blue},
  filecolor={Maroon},
  citecolor={Blue},
  urlcolor={Blue},
  pdfcreator={LaTeX via pandoc}}

\title{Computational methods in landscape ecology}


  \author{Maximilian H.K. Hesselbarth}
            \affil{%
                  International Institute for Applied Systems Analysis,
                  Biodiversity, Ecology, and Conservation Group,
                  Laxenburg, Austria
              }
        \author{Jakub Nowosad}
            \affil{%
                  Adam Mickiewicz University, Poznan, Poland
              }
        \author{Marti Bosch}
            \affil{%
                  École polytechnique fédérale de Lausanne, Lausanne,
                  Switzerland
              }
        \author{Martin Jung}
            \affil{%
                  International Institute for Applied Systems Analysis,
                  Biodiversity, Ecology, and Conservation Group,
                  Laxenburg, Austria
              }
        \author{Rafael Schouten}
            \affil{%
                  Globe Institute, University of Copenhagen, Copenhagen,
                  Denmark
              }
      
\date{}
\begin{document}
\maketitle

\setstretch{1.5}
\section{Introduction}\label{introduction}

Landscapes are typically defined as mosaics of different land covers,
habitats, ecosystems, or land use systems (Forman and Godron 1986;
Forman 1995) - with empathize on heterogeneity of at least one factor of
interest (Turner and Gardner 2015). Linking spatial heterogeneity and
ecological processes, including potential feedbacks, is the fundamental
concept of landscape ecology (Risser et al. 1984; Turner 1989, 2005).
Typical research topics include ecological flows in landscape mosaics,
drivers and consequences of land use and land cover change, nonlinear
dynamics and landscape complexity, scale and scaling, spatial analysis
and landscape modeling, linking landscape metrics to ecological
processes, integrating humans and their activities into landscape
ecology, landscape conservation and sustainability, or accuracy
assessment and uncertainty analysis (Wu and Hobbs 2002; Wu 2013).

Computational science involves analyzing abstracted core mechanisms of
research questions using data and algorithms. Thus, computational
ecology can be defined as computational science that is used to address
ecological research questions with focus on complex, adaptive systems
and data-driven approaches (Poisot et al. 2019). Computational ecology
in general and simulations specifically are of importance for ecology
because ecological data is often context- and scale-dependent and
unfeasible to study in controllable, reproducible, and replicable
experimental settings (Petrovskii and Petrovskaya 2012; but see Wiersma
2022). Because landscape ecology is a cross-disciplinary field
including, e.g., social sciences, geography, or ecology and evolution
(Wiens 1997; With 2019), the general increase of available data (Chi et
al. 2016; Jarić et al. 2020; Wüest et al. 2020; Nathan et al. 2022), or
the complex landscape systems (Newman et al. 2019), there is the need
for sophisticated computational methods in order to link heterogeneity
patterns and ecological processes. Thus, computational methods are one
of the most important tools of modern scientific research (Prlić and
Procter 2012; Wilson et al. 2014).

Here, we aim to introduce the latest developments of computational
methods in landscape ecology. However, we do not provide a systematic
literature review, or a general introduction to (computational)
landscape ecology. For a more comprehensive, general introductions to
landscape ecology, please see Turner and Gardner (2015), or Gergel and
Turner (2017). First, we will introduce recent computational
developments of the most important topics of landscape ecology. This
includes, for example, spatial patterns quantification, connectivity and
movement, or model simulations. Second, we will introduce common
software tools that implement or are potentially capable of implementing
these developments. However, due to its many advantages, we will focus
on open-source software (von Krogh and von Hippel 2006), and
specifically scripting languages, i.e., R, Python, and Julia. Last, we
will discuss potential future needs developments of computational
methods in landscape ecology..

\section{Simulation models}\label{simulation-models}

Simulation models are a powerful tool to study complex adaptive
ecological systems in controllable, reproducible, and replicable
settings (Pascual 2005; Petrovskii and Petrovskaya 2012). Thus, they can
be seen as experimental systems that allow, unlike to nature, all
imaginable manipulations in order to advance theoretical developments or
test hypotheses (Peck 2004). Due to the general temporal and spatial
scales, complex interactions and feedbacks, or scale mismatches between
patterns and processes, simulation models are one of the major
approaches in landscape ecology (Schröder and Seppelt 2006; Synes et al.
2016). In general, simulations models can be classified using two major
divisions, namely as \emph{i)} predictive or exploratory models, and
\emph{ii)} pattern- or process-based models (Peck 2004; Synes et al.
2016). Here, we are going to focus mainly on exploratory models, but
include both pattern- and process-based models.

\subsection{Pattern based}\label{pattern-based}

\subsubsection{Neutral landscapes
models}\label{neutral-landscapes-models}

Neutral landscapes models simulate landscape patterns without assuming
any underlying abiotic or biotic processes. Thus, they are often used as
null hypothesis or baselines for comparisons with real landscape
patterns, while controlling certain aspects of the landscape (Li et al.
2004; Wang and Malanson 2008). The first neutral models simply assigned
land cover types randomly to cells in the landscape and were based on
percolation theory. Further developments included hierarchical models
that considered different spatial scales while assigning land cover
types. While these first two classes can only simulate landscape with
discrete land cover classes, fractal neutral models can also simulate
continuous land cover maps by utilizing e.g., Brownian motion. Last,
more complex neutral models were developed that are based on contagion,
autocorrelation, or spatial aggregation concepts to simulate more
realistic landscapes (Wang and Malanson 2008). In order to make multiple
neutral landscape model algorithms available several software libraries
are available in prominent programming languages including Python
{[}NLMpy; Etherington et al. (2015){]} or R {[}NLMR; Sciaini et al.
(2018){]}.

\subsubsection{Cellular Automata}\label{cellular-automata}

\subsection{Process based}\label{process-based}

\href{https://doi.org/10.1111/ecog.05687}{RangeShifter}

\subsubsection{Landscape Generator}\label{landscape-generator}

\href{https://doi.org/10.1371/journal.pone.0064968}{G-RaFFE}
\href{https://doi.org/10.1002/ece3.2145}{LG}

\subsubsection{Coupled lattice}\label{coupled-lattice}

\subsubsection{IBMs/ABMs}\label{ibmsabms}

\subsection{Misc}\label{misc}

\href{https://doi.org/10.1016/j.ecolmodel.2019.108837}{Marine SDMS}
\href{https://doi.org/10.1111/jbi.14617}{SDM past present future}

\href{https://doi.org/10.1111/j.1365-294X.2010.04678.x}{Simulations in
landscape genetics}
\href{https://doi.org/10.1007/s10980-017-0605-9}{HexSim modeling
environment} \href{https://doi.org/10.1016/j.tree.2023.04.010}{Digital
twins?}

\newpage{}

\subsubsection{References}\label{references}

\phantomsection\label{refs}
\begin{CSLReferences}{1}{1}
\bibitem[\citeproctext]{ref-Chi2016}
Chi M, Plaza A, Benediktsson JA, et al (2016) Big data for remote
sensing: Challenges and opportunities. Proceedings of the IEEE
104:2207--2219. \url{https://doi.org/10.1109/JPROC.2016.2598228}

\bibitem[\citeproctext]{ref-Etherington2015}
Etherington TR, Holland EP, O'Sullivan D (2015) NLMpy: A python software
package for the creation of neutral landscape models within a general
numerical framework. Methods in Ecology and Evolution 6:164--168.
\url{https://doi.org/10.1111/2041-210X.12308}

\bibitem[\citeproctext]{ref-Forman1995}
Forman RTT (1995) Land mosaics: The ecology of landscapes and regions.
Cambridge University Press, Cambridge, UK

\bibitem[\citeproctext]{ref-Forman1986}
Forman RTT, Godron M (1986) Landscape ecology. Wiley \& Sons,
Chichester, UK

\bibitem[\citeproctext]{ref-Gergel2017}
Gergel SE, Turner MG (eds) (2017)
\href{https://doi.org/10.1007/978-1-4939-6374-4}{Learning landscape
ecology}. Springer New York, New York, NY

\bibitem[\citeproctext]{ref-Jaric2020}
Jarić I, Correia RA, Brook BW, et al (2020) iEcology: Harnessing large
online resources to generate ecological insights. Trends in Ecology \&
Evolution 35:630--639. \url{https://doi.org/10.1016/j.tree.2020.03.003}

\bibitem[\citeproctext]{ref-Li2004}
Li X, He HS, Wang X, et al (2004) Evaluating the effectiveness of
neutral landscape models to represent a real landscape. Landscape and
Urban Planning 69:137--148.
\url{https://doi.org/10.1016/j.landurbplan.2003.10.037}

\bibitem[\citeproctext]{ref-Nathan2022}
Nathan R, Monk CT, Arlinghaus R, et al (2022) Big-data approaches lead
to an increased understanding of the ecology of animal movement. Science
375:eabg1780. \url{https://doi.org/10.1126/science.abg1780}

\bibitem[\citeproctext]{ref-Newman2019}
Newman EA, Kennedy MC, Falk DA, McKenzie D (2019) Scaling and complexity
in landscape ecology. Frontiers in Ecology and Evolution 7:293.
\url{https://doi.org/10.3389/fevo.2019.00293}

\bibitem[\citeproctext]{ref-Pascual2005}
Pascual M (2005) Computational ecology: From the complex to the simple
and back. PLoS Computional Biology 1:e18.
\url{https://doi.org/10.1371/journal.pcbi.0010018}

\bibitem[\citeproctext]{ref-Peck2004}
Peck SL (2004) Simulation as experiment: A philosophical reassessment
for biological modeling. Trends in Ecology \& Evolution 19:530--534.
\url{https://doi.org/10.1016/j.tree.2004.07.019}

\bibitem[\citeproctext]{ref-Petrovskii2012}
Petrovskii S, Petrovskaya N (2012) Computational ecology as an emerging
science. Interface Focus 2:241--254.
\url{https://doi.org/10.1098/rsfs.2011.0083}

\bibitem[\citeproctext]{ref-Poisot2019}
Poisot T, LaBrie R, Larson E, et al (2019) Data-based, synthesis-driven:
Setting the agenda for computational ecology. Ideas in Ecology and
Evolution 12: \url{https://doi.org/10.24908/iee.2019.12.2.e}

\bibitem[\citeproctext]{ref-Prlic2012}
Prlić A, Procter JB (2012) Ten simple rules for the open development of
scientific software. PLoS Computational Biology 8:e1002802.
\url{https://doi.org/10.1371/journal.pcbi.1002802}

\bibitem[\citeproctext]{ref-Risser1984}
Risser PG, Karr JR, Forman RTT (1984) Landscape ecology: Directions and
approaches. Illinois Natural History Survery Special Publication 2:7--14

\bibitem[\citeproctext]{ref-Schroder2006}
Schröder B, Seppelt R (2006) Analysis of pattern--process interactions
based on landscape models - overview, general concepts, and
methodological issues. Ecological Modelling 199:505--516.
\url{https://doi.org/10.1016/j.ecolmodel.2006.05.036}

\bibitem[\citeproctext]{ref-Sciaini2018}
Sciaini M, Fritsch M, Scherer C, Simpkins CE (2018) NLMR and
landscapetools: An integrated environment for simulating and modifying
neutral landscape models in r. Methods in Ecology and Evolution
9:2240--2248. \url{https://doi.org/10.1111/2041-210X.13076}

\bibitem[\citeproctext]{ref-Synes2016}
Synes NW, Brown C, Watts K, et al (2016) Emerging opportunities for
landscape ecological modelling. Current Landscape Ecology Reports
1:146--167

\bibitem[\citeproctext]{ref-Turner1989}
Turner MG (1989) Landscape ecology: The effect of pattern on process.
Annual Review of Ecology and Systematics 20:171--197.
\url{https://doi.org/10.1146/annurev.es.20.110189.001131}

\bibitem[\citeproctext]{ref-Turner2005}
Turner MG (2005) Landscape ecology: What is the state of the science?
Annual Review of Ecology, Evolution, and Systematics 36:319--344.
\url{https://doi.org/10.1146/annurev.ecolsys.36.102003.152614}

\bibitem[\citeproctext]{ref-Turner2015}
Turner MG, Gardner RH (2015) Landscape ecology in theory and practice:
Pattern and process, 2nd edition. Springer, New York

\bibitem[\citeproctext]{ref-vonKrogh2006}
von Krogh G, von Hippel E (2006) The promise of research on open source
software. Management Science 52:975--983.
\url{https://doi.org/10.1287/mnsc.1060.0560}

\bibitem[\citeproctext]{ref-Wang2008}
Wang Q, Malanson GP (2008) Neutral landscapes: Bases for exploration in
landscape ecology. Geography Compass 2:319--339.
\url{https://doi.org/10.1111/j.1749-8198.2008.00090.x}

\bibitem[\citeproctext]{ref-Wiens1997}
Wiens JA (1997)
\href{https://doi.org/10.1016/B978-012323445-2/50005-5}{Metapopulation
dynamics and landscape ecology}. In: Metapopulation biology. Elsevier,
pp 43--62

\bibitem[\citeproctext]{ref-Wiersma2022}
Wiersma YF (2022) A review of landscape ecology experiments to
understand ecological processes. Ecological Processes 11:57.
\url{https://doi.org/10.1186/s13717-022-00401-0}

\bibitem[\citeproctext]{ref-Wilson2014}
Wilson G, Aruliah DA, Brown CT, et al (2014) Best practices for
scientific computing. PLoS Biology 12:e1001745.
\url{https://doi.org/10.1371/journal.pbio.1001745}

\bibitem[\citeproctext]{ref-With2019}
With KA (2019) Essentials of landscape ecology, 1st edn. Oxford
University Press, Oxford, UK

\bibitem[\citeproctext]{ref-Wu2013}
Wu J (2013) Key concepts and research topics in landscape ecology
revisited: 30 years after the allerton park workshop. Landscape Ecology
28:1--11. \url{https://doi.org/10.1007/s10980-012-9836-y}

\bibitem[\citeproctext]{ref-Wu2002}
Wu J, Hobbs R (2002) Key issues and research priorities in landscape
ecology: An idiosyncratic synthesis. Landscape Ecology 17:355--365.
\url{https://doi.org/10.1023/A:1020561630963}

\bibitem[\citeproctext]{ref-Wuest2020}
Wüest RO, Zimmermann NE, Zurell D, et al (2020) Macroecology in the age
of big data -- where to go from here? Journal of Biogeography 47:1--12.
\url{https://doi.org/10.1111/jbi.13633}

\end{CSLReferences}



\end{document}
